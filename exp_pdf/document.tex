\documentclass[12pt,a4paper]{article}

% =====================
% Encoding & Fonts
% =====================
\usepackage[utf8]{inputenc}
\usepackage[T1]{fontenc}
\usepackage{newtxtext,newtxmath} % professional serif + math

% =====================
% Layout & Spacing
% =====================
\usepackage{geometry}
\geometry{a4paper, margin=1in}
\usepackage{setspace}
\onehalfspacing
\setlength{\parskip}{6pt}
\usepackage{microtype}
\usepackage{parskip}

% =====================
% Colors, Links, Headers
% =====================
\usepackage{xcolor}
\definecolor{accent}{RGB}{0,63,114} % deep navy
\definecolor{lightgray}{RGB}{245,245,245}
\usepackage{hyperref}
\hypersetup{
  colorlinks=true,
  linkcolor=accent,
  urlcolor=accent,
  citecolor=accent
}
\usepackage{fancyhdr}
\pagestyle{fancy}
\fancyhf{}
\fancyhead[L]{Experimental Report: Deployment and Execution of Java Servlets}
\fancyhead[R]{\thepage}
\renewcommand{\headrulewidth}{0.4pt}

% =====================
% Section Styling
% =====================
\usepackage{titlesec}
\titleformat{\section}{\large\bfseries\color{accent}}{\thesection}{1em}{}
\titleformat{\subsection}{\normalsize\bfseries\color{accent}}{\thesubsection}{1em}{}
\titleformat{\subsubsection}{\normalsize\bfseries}{\thesubsubsection}{1em}{}

% =====================
% Figures, Tables, Code
% =====================
\usepackage{graphicx}
\graphicspath{{./}}
\usepackage{float}
\usepackage{booktabs}
\usepackage{array}
\usepackage{caption}
\captionsetup{font=small, labelfont=bf, labelsep=colon}
\usepackage{listings}
\lstset{
  backgroundcolor=\color{lightgray},
  basicstyle=\ttfamily\footnotesize,
  frame=single,
  breaklines=true,
  showstringspaces=false,
  columns=fullflexible,
  tabsize=2,
  keepspaces=true
}
\usepackage{enumitem}
\setlist[itemize]{noitemsep, topsep=0pt}
\setlist[enumerate]{itemsep=4pt, topsep=4pt}

% =====================
% Title Meta
% =====================
\title{\vspace{-1cm}\bfseries\LARGE Experimental Report\\[4pt]\Large\color{accent}Deployment and Execution of Java Servlets}
\author{\normalsize Mohammad Abdullah\\\normalsize Student ID: 202322240356\\\normalsize Course: Java Web}
\date{13 October 2025}

\begin{document}

% ---------- Title Page ----------
\begin{titlepage}
    \centering
    \vspace*{1.5cm}
    {\Huge\bfseries Experimental Report\\[6pt]}
    {\Large\color{accent}\bfseries Deployment and Execution of Java Servlets\par}
    \vspace{1.5cm}
    \begin{tabular}{@{}ll@{}}
        \textbf{Student Name:} & Mohammad Abdullah \\
        \textbf{Student ID:} & 202322240356 \\
        \textbf{Course:} & Java Web \\
        \textbf{Date:} & 13 October 2025 \\
    \end{tabular}
    \vfill
    % Optional logo: replace with your institution image file
    % \includegraphics[width=0.28\textwidth]{institution-logo.png}\\[0.4cm]
    {\large Department of Computer Science\\China West Normal University}
\end{titlepage}

% ---------- Abstract & Keywords ----------
\begin{abstract}
\noindent
This experiment configures, deploys, and executes Java Servlets within an Apache Tomcat 11 environment using Visual Studio Code and Maven. It addresses Jakarta EE migration from legacy Java EE namespaces, implements a reproducible build and deployment pipeline, and validates the deployment via structured functional testing. Results demonstrate full compatibility with Jakarta Servlet API 6 and a reliable manual WAR deployment workflow.
\end{abstract}

\noindent\textbf{Keywords:} Java Servlets; Tomcat 11; Jakarta EE; Maven; VS Code; Deployment; Web.xml

% ---------- 1. Objective ----------
\section{Objective}
The primary objective of this experiment was to configure, deploy, and execute Java Servlets in an Apache Tomcat~11 environment using Visual Studio Code. The work focused on resolving compatibility issues between Jakarta EE and legacy Java EE specifications, packaging the application using Maven, and validating the deployment through comprehensive testing.

% ---------- 2. Environment Setup & Tools ----------
\section{Environment Setup and Tools}
The following software components were used:

\begin{table}[H]
\centering
\begin{tabular}{@{}ll@{}}
\toprule
\textbf{Component} & \textbf{Specification} \\
\midrule
Operating System & Windows 11 \\
Integrated Development Environment & Visual Studio Code (Version 1.94) \\
Java Development Kit & Eclipse Adoptium JDK 17.0.15 \\
Application Server & Apache Tomcat 11.0.0-M6 \\
Build Automation Tool & Apache Maven 3.9.11 \\
\bottomrule
\end{tabular}
\caption{Software environment specifications.}
\end{table}

\noindent\textbf{Key VS Code Extensions:}
\begin{itemize}
    \item Extension Pack for Java (Microsoft)
    \item Community Server Connectors (Red Hat)
    \item Maven for Java
\end{itemize}

% ---------- 3. Configuration & Deployment Procedure ----------
\section{Configuration and Deployment Procedure}

\subsection{Project Initialization and Structure}
A new Maven web project was created using the \texttt{maven-archetype-webapp} archetype. The project structure follows Maven conventions, ensuring clear separation of Java sources, web resources, and configuration files.

\begin{figure}[H]
    \centering
    \includegraphics[height=0.4\textwidth]{project-structure.png}
    \caption{VS Code Explorer showing the project structure.}
    \label{fig:project-structure}
\end{figure}

\subsection{Jakarta EE Compatibility Resolution}
\label{subsec:jakarta-resolution}
Initial deployment attempts resulted in HTTP~500 errors due to legacy \texttt{javax.servlet} imports being incompatible with Tomcat~11.

\noindent\textbf{Problem:} Tomcat~11 requires Jakarta EE packages, whereas the code used deprecated \texttt{javax.servlet} packages.

\noindent\textbf{Solution:} Update all servlet imports to the Jakarta namespace (\texttt{jakarta.servlet}).

\begin{lstlisting}[language=Java, caption={Namespace migration from Java EE to Jakarta EE.}]
// BEFORE (Java EE - incompatible with Tomcat 11)
import javax.servlet.ServletException;
import javax.servlet.http.HttpServlet;

// AFTER (Jakarta EE - compatible with Tomcat 11)
import jakarta.servlet.ServletException;
import jakarta.servlet.http.HttpServlet;
\end{lstlisting}

\subsection{Maven Dependencies Configuration}
The \texttt{pom.xml} was configured with Jakarta EE dependencies to ensure compatibility with Tomcat~11:

\begin{lstlisting}[language=XML, caption={Relevant dependencies in pom.xml}]
<dependencies>
  <dependency>
    <groupId>jakarta.servlet</groupId>
    <artifactId>jakarta.servlet-api</artifactId>
    <version>6.0.0</version>
    <scope>provided</scope>
  </dependency>
  <dependency>
    <groupId>jakarta.servlet.jsp</groupId>
    <artifactId>jakarta.servlet.jsp-api</artifactId>
    <version>3.1.1</version>
    <scope>provided</scope>
  </dependency>
</dependencies>
\end{lstlisting}

\subsection{Web Application Descriptor}
The \texttt{web.xml} was updated with the Jakarta EE namespace and servlet mappings:

\begin{lstlisting}[language=XML, caption={web.xml with Jakarta EE namespace and servlet mapping}]
<?xml version="1.0" encoding="UTF-8"?>
<web-app xmlns="https://jakarta.ee/xml/ns/jakartaee"
         xmlns:xsi="http://www.w3.org/2001/XMLSchema-instance"
         xsi:schemaLocation="https://jakarta.ee/xml/ns/jakartaee 
         https://jakarta.ee/xml/ns/jakartaee/web-app_6_0.xsd"
         version="6.0">
    
    <servlet>
        <servlet-name>TestServlet01</servlet-name>
        <servlet-class>com.example.servlets.TestServlet01</servlet-class>
    </servlet>
    
    <servlet-mapping>
        <servlet-name>TestServlet01</servlet-name>
        <url-pattern>/TestServlet01</url-pattern>
    </servlet-mapping>
</web-app>
\end{lstlisting}

\subsection{Build and Deployment Process}
The application underwent a systematic build and deployment process:

\begin{enumerate}
  \item \textbf{Maven Build Execution:}
\begin{lstlisting}[language=bash]
mvn clean package
\end{lstlisting}

  \item \textbf{Manual Deployment Steps:}
  \begin{itemize}
    \item Copy the WAR file (\texttt{ServletDemo.war}) to Tomcat's \texttt{webapps/} directory.
    \item Restart Tomcat from the VS Code Servers view.
    \item Verify automatic extraction by checking the creation of the \texttt{ServletDemo/} folder.
  \end{itemize}
\end{enumerate}

\begin{figure}[H]
    \centering
    \includegraphics[width=0.8\textwidth]{maven-build.png}
    \caption{VS Code terminal showing a successful Maven build.}
    \label{fig:maven-build}
\end{figure}

\begin{figure}[H]
    \centering
    \includegraphics[width=0.6\textwidth]{tomcat-server.png}
    \caption{VS Code Servers view showing Tomcat~11 running.}
    \label{fig:tomcat-server}
\end{figure}

% ---------- 4. Testing and Results ----------
\section{Testing and Results}

\subsection{Test Execution Summary}
Deployment was validated via systematic testing of all application endpoints.

\begin{table}[H]
\centering
\begin{tabular}{@{}p{0.25\textwidth} p{0.35\textwidth} p{0.2\textwidth} p{0.1\textwidth}@{}}
\toprule
\textbf{Test Case} & \textbf{URL} & \textbf{Expected Result} & \textbf{Status} \\
\midrule
TestServlet01 & \texttt{/ServletDemo/TestServlet01} & "Hello Mohammad Abdullah" & PASS \\
TestServlet & \texttt{/ServletDemo/test} & HTML confirmation page & PASS \\
Application Root & \texttt{/ServletDemo/} & Context accessibility & PASS \\
ServletDemo App & \texttt{/ServletDemo/} & Full application interface & PASS \\
\bottomrule
\end{tabular}
\caption{Test execution summary.}
\label{tab:test-summary}
\end{table}

\subsection{Test Case Details}

\subsubsection{TestServlet01 Validation}
\begin{lstlisting}[language=bash]
URL: http://localhost:8080/ServletDemo/TestServlet01
HTTP Method: GET
Response Code: 200 OK
Content-Type: text/html
Output: "Hello Mohammad Abdullah"
\end{lstlisting}

\textbf{Success Criteria Met:}
\begin{itemize}
  \item Servlet instantiation and initialization
  \item HTTP request processing
  \item Dynamic content generation
  \item Proper character encoding
\end{itemize}

\begin{figure}[H]
    \centering
    \includegraphics[height=0.4\textwidth]{testservlet01-output.png}
    \caption{Successful execution of \texttt{TestServlet01}.}
    \label{fig:testservlet01-output}
\end{figure}

\subsubsection{Application Interface Validation}
\begin{lstlisting}[language=bash]
URL: http://localhost:8080/ServletDemo/
HTTP Method: GET
Response Code: 200 OK
Content-Type: text/html
Output: Complete application interface
\end{lstlisting}

\textbf{Success Criteria Met:}
\begin{itemize}
  \item Application context accessibility
  \item Proper resource loading
  \item Navigation functionality
  \item User interface presentation
\end{itemize}

\begin{figure}[H]
    \centering
\includegraphics[height=0.4\textwidth]{ServletDemo Application.png}
    \caption{Successful execution of the ServletDemo application.}
    \label{fig:servletdemo-application}
\end{figure}

\subsubsection{TestServlet Validation}
\begin{lstlisting}[language=bash]
URL: http://localhost:8080/ServletDemo/test
HTTP Method: GET
Response Code: 200 OK
Content-Type: text/html
Output: HTML page with confirmation message
\end{lstlisting}

\textbf{Success Criteria Met:}
\begin{itemize}
  \item HTML content generation
  \item Multiple servlet deployment
  \item URL pattern mapping
  \item Container resource management
\end{itemize}

\begin{figure}[H]
    \centering
    \includegraphics[height=0.4\textwidth]{testservlet-output.png}
    \caption{Successful execution of \texttt{TestServlet}.}
    \label{fig:testservlet-output}
\end{figure}

\subsection{Technical Validation}
Testing confirmed the following:
\begin{itemize}
    \item \textbf{Jakarta EE Compatibility:} Namespace conflicts resolved; code runs under Servlet API~6.
    \item \textbf{Deployment Process:} WAR deployment and automatic extraction verified.
    \item \textbf{Container Integration:} Proper integration with Tomcat~11 servlet container.
    \item \textbf{Build Process:} Maven compilation and packaging functioning as expected.
    \item \textbf{User Interface:} Complete application accessibility and navigation.
\end{itemize}

% ---------- 5. Problems Encountered and Resolutions ----------
\section{Problems Encountered and Resolutions}

\subsection{Technical Challenges Overview}

\begin{table}[H]
\centering
\begin{tabular}{@{}p{0.25\textwidth} p{0.3\textwidth} p{0.35\textwidth}@{}}
\toprule
\textbf{Problem} & \textbf{Symptom} & \textbf{Resolution} \\
\midrule
Servlet Instantiation Error & HTTP 500 -- ClassNotFoundException & Migrated from \texttt{javax} to \texttt{jakarta} namespace \\
Deployment Failure & Application context not available & Implemented manual WAR deployment \\
Build Configuration & Maven compilation failures & Updated dependencies and Java version \\
Context Accessibility & HTTP 404 on application root & Verified deployment and server restart \\
\bottomrule
\end{tabular}
\caption{Problem resolution summary.}
\label{tab:problem-resolution}
\end{table}

\subsection{Detailed Problem Analysis}

\subsubsection{Jakarta EE Compatibility Issue}
\textbf{Problem:} Servlet instantiation failure due to namespace conflicts between Java EE and Jakarta EE.

\textbf{Technical Details:}
\begin{lstlisting}[language=bash]
Root Cause: java.lang.NoClassDefFoundError: javax/servlet/http/HttpServlet
Impact: Complete servlet functionality failure
Environment: Tomcat 11.0.0-M6 (Jakarta EE) vs Java EE code
\end{lstlisting}

\textbf{Solution Implementation:}
\begin{enumerate}
  \item Updated all servlet import statements.
  \item Modified Maven dependencies to Jakarta EE 6.0.0.
  \item Updated web.xml namespace declaration.
  \item Recompiled and redeployed the application.
\end{enumerate}

\textbf{Outcome:} Successful servlet execution returning HTTP~200 responses.

\subsubsection{Deployment Process Challenges}
\textbf{Problem:} Automated deployment through VS Code Server Connectors was unreliable.

\textbf{Technical Details:}
\begin{lstlisting}[language=bash]
Symptoms: WAR file not deployed, context path unavailable
Constraints: IDE extension limitations, permission issues
Impact: Manual intervention required for deployment
\end{lstlisting}

\textbf{Solution Implementation:}
\begin{enumerate}
  \item Established a manual deployment workflow.
  \item Implemented deployment verification checks.
  \item Used VS Code for server lifecycle management only.
  \item Maintained a consistent deployment procedure.
\end{enumerate}

\textbf{Outcome:} Reliable and repeatable deployment process.

\subsubsection{Build System Configuration}
\textbf{Problem:} Maven build failures due to dependency and configuration issues.

\textbf{Technical Details:}
\begin{lstlisting}[language=bash]
Error: Compilation failure - package does not exist
Cause: Incorrect dependency scope and version
Impact: Build process interruption
\end{lstlisting}

\textbf{Solution Implementation:}
\begin{enumerate}
  \item Corrected dependency scope to \texttt{provided}.
  \item Updated Java version compatibility.
  \item Implemented clean build practices.
  \item Verified dependency resolution.
\end{enumerate}

\textbf{Outcome:} Consistent, successful builds with proper artifact generation.

\begin{figure}[H]
    \centering
    \includegraphics[width=0.8\textwidth]{troubleshooting-flowchart.png}
    \caption{Systematic troubleshooting approach for deployment issues.}
    \label{fig:troubleshooting-flowchart}
\end{figure}

\subsection{Preventive Measures}
To avoid similar issues in future projects:
\begin{itemize}
  \item \textbf{Environment Validation:} Verify container specifications before development.
  \item \textbf{Dependency Management:} Use compatible dependency versions from project inception.
  \item \textbf{Deployment Automation:} Establish reliable deployment scripts and procedures.
  \item \textbf{Testing Strategy:} Implement comprehensive testing at each deployment stage.
  \item \textbf{Documentation:} Maintain updated technical documentation for reference.
\end{itemize}

\subsection{Resolution Effectiveness}
All implemented resolutions were effective with the following outcomes:
\begin{itemize}
  \item 100\% resolution of critical deployment blockers.
  \item Consistent application availability post-resolution.
  \item No recurrence of resolved issues across multiple deployment cycles.
  \item Improved deployment reliability and predictability.
\end{itemize}

% ---------- 6. Conclusion ----------
\section{Conclusion}

\subsection{Experimental Summary}
This experiment successfully achieved its primary objective of configuring, deploying, and executing Java Servlets in a Tomcat~11 environment using Visual Studio Code. Testing confirmed complete functionality of the deployed servlets and robustness of the development environment.

\subsection{Key Findings}
\begin{itemize}
    \item \textbf{Environment Integration:} VS Code with essential extensions provides an effective environment for Java web applications.
    \item \textbf{Specification Compatibility:} Jakarta EE migration resolved container compatibility requirements.
    \item \textbf{Deployment Reliability:} Manual WAR deployment proved more reliable than automated IDE deployment in this context.
    \item \textbf{Development Efficiency:} Maven ensures consistent and reproducible packaging.
\end{itemize}

\subsection{Technical Validation}
\begin{table}[H]
\centering
\begin{tabular}{@{}p{0.7\textwidth} p{0.2\textwidth}@{}}
\toprule
\textbf{Technical Aspect} & \textbf{Status} \\
\midrule
Servlet API Implementation & Verified \\
HTTP Protocol Compliance & Verified \\
Container Integration & Verified \\
Build Process Reliability & Verified \\
Deployment Consistency & Verified \\
Performance Requirements & Verified \\
\bottomrule
\end{tabular}
\caption{Technical validation results.}
\label{tab:technical-validation}
\end{table}

\subsection{Practical Implications}
\begin{enumerate}
  \item \textbf{Development Workflow:} A reproducible development and deployment workflow was established.
  \item \textbf{Troubleshooting Methodology:} Effective approaches for common deployment issues were demonstrated.
  \item \textbf{Toolchain Integration:} Modern tooling for enterprise Java projects was validated.
  \item \textbf{Knowledge Transfer:} The procedures documented here support future project implementations.
\end{enumerate}

\subsection{Conclusion Statement}
\begin{quote}
A properly configured environment comprising Visual Studio Code, Apache Maven, and Apache Tomcat~11 provides a robust platform for Java Servlet development. Systematic resolution of technical challenges, particularly the Jakarta EE migration, enables sustainable and maintainable web application practices.
\end{quote}

\end{document}
